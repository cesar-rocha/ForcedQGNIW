\documentclass[12pt]{article}
\usepackage{fullpage}
\usepackage{lineno}
\usepackage[notcite,notref]{showkeys}
\usepackage[notcite,notref]{showkeys}
\usepackage{amssymb}
\usepackage{amsmath}
\usepackage{natbib}

\usepackage{epsfig}
\usepackage[mathscr]{eucal}



\bibliographystyle{plain}

%% For lucida bright
%\usepackage[T1]{fontenc}
%\usepackage{lucidabr}
%\usepackage{bm}
%%%




\usepackage{color,amssymb,amsmath,amsthm}

\usepackage{epsfig}
\usepackage[mathscr]{eucal}


\newcommand{\NIW}{near-inertial wave}
\newcommand{\macrot}{macroturbulence}

\include{symbols}

%\renewcommand{\sZ}{\mathsf{Z}}
%\renewcommand{\sE}{\mathsf{E}}
%\newcommand{\iBu}{\left(\tfrac{f_0}{N}\right)^2}
\newcommand{\F}{\mathcal{F}}
\newcommand{\D}{\mathcal{D}}
\newcommand{\phis}{\phi^\star}

%\newcommand{\bk}{\boldsymbol{k}}

\newcommand{\cg}{\mathbf{c}_g}
\newcommand{\Uf}{\mathbf{U}}
\renewcommand{\Im}{\mathrm{Im}}
\renewcommand{\div}{\nabla\cdot}
\renewcommand{\P}{\mathcal{P}}
\newcommand{\dU}{\delta U}
\newcommand{\W}{\mathcal{W}}
\newcommand{\cK}{\mathcal{K}}
\newcommand{\cP}{\mathcal{P}}
\renewcommand{\L}{\mathsf{L}}
\renewcommand{\N}{\mathsf{N}}
\newcommand{\psiq}{\psi^q}
\newcommand{\psiw}{\psi^w}
%\newcommand{\tfrac}{\frac}
%\newcommand{\eqref}{\ref}
\newcommand{\kb}{\mathbf{k}}
\newcommand{\xb}{\mathbf{x}}
%wave PV
\newcommand{\qw}{q^{\mathrm{w}}}
\newcommand{\bw}{b^{\mathrm{w}}}
\newcommand{\ug}{u^{\mathrm{g}}}
\newcommand{\bug}{\bu^{\mathrm{g}}}

%action and energy
\newcommand{\A}{  \mathcal{A}}
\newcommand{\E}{\mathcal{E}}
\newcommand{\Pw}{\mathcal{P}}
\newcommand{\Ke}{\mathcal{K}}
\newcommand{\Ff}{ \boldsymbol{\mathcal{F}}}
\newcommand{\Ffp}{\boldsymbol{\mathcal{F}}^{\perp}}
\newcommand{\Hf}{\boldsymbol{\mathcal{H}}}
\newcommand{\Gg}{\boldsymbol{\mathcal{G}}}
\newcommand{\epA}{\varepsilon_\mathcal{A}}
\newcommand{\epP}{\varepsilon_\mathcal{P}}
\newcommand{\epK}{\varepsilon_\mathcal{K}}

%dispersivity
\newcommand{\disp}{\eta}

%relative vorticity
\newcommand{\ze}{\zeta}

% index differentiation
\newcommand{\gind}[2]{#1_{,#2}}

% QG-NIW model (or XV model)
\newcommand{\coupledmodel}{QG-NIW model}

% expectation
\newcommand{\Ep}{\mathbb{E}}

\begin{document}





\title{Equilibration of forced barotropic turbulence by stimulated generation
of near-inertial waves}

\author{
CR  \& WRY \thanks {Scripps Institution of Oceanography,
University of California at San Diego, La Jolla, CA
92093--0230, USA.
%\protect\url{email:wryoung@ucsd.edu}.
}
}


\maketitle

\section{The model}
The balanced dynamics satisfies
\beq
q_t + \sJ(\psi,q)  = \xi_q -\mu \zeta + \D_q \com
\label{balanced_dynamics}
\eeq
where the quasigeostrophic potential vorticity is
\beq
\label{qgpv}
q = \underbrace{\lap \psi}_{\defn \zeta} +
                \underbrace{\tfrac{1}{f_0}\Big[ \tfrac{1}{4} \lap |\phi|^2 + \tfrac{\ii}{2}
                \sJ(\phi^\star,\phi)\Big]}_{\defn q^w}\per
\eeq
The back-rotated near-inertial velocity $\phi$ satisfies the
YBJ equation
\beq
\phi_t + \sJ(\psi,\phi) +  \phi \tfrac{\ii}{2}\ze - \tfrac{\ii}{2} \disp \lap \phi
 = \xi_\phi -\gamma \phi + \D_\phi\per
 \label{ybj_dynamics}
\eeq

In \eqref{balanced_dynamics}, $\mu$ is the linear drag, $\xi_q$ is a white-noise forcing
with random horizontal structure with a ring-like isotropic spectrum peaking at $|\mathbf{k}_f|$
and variance $\sigma^2_q$.
In \eqref{ybj_dynamics}, $\xi_\phi$ is a white-noise spatially uniform forcing with
variance $\sigma^2_\phi$, and $\gamma$
is a linear damping. In \eqref{balanced_dynamics} and \eqref{ybj_dynamics}, the $\D$ terms represent
small-scale dissipation.

\subsection{Power integrals}

The balanced kinetic energy equation is
\beq
\frac{\dd}{\dd t} \underbrace{\half \la |\nabla \psi|^2 \ra}_{\defn \K} = -\la\Gamma_r + \Gamma_a\ra + \Xi +
 -\la \psi \xi_q \ra -\ \mu \la|\nabla\psi|^2\ra - \la\psi\D_q\ra\com
\label{Ke}
\eeq
where $\Gamma_r$ and $\Gamma_a$ are energy conversion terms and $\Xi$ is a source of balanced kinetic energy
due to wave dissipation (cf. RWY); $\la \ra$ represents spatial average.
If the were no waves, i.e., $\phi(t=0)=0$ and $\sigma_\phi^2=0$,  then $\Gamma_r=\Gamma_a = 0$, and the expectation of the work due to the white-noise forcing
is $-\la\psi\xi_q\ra = \sigma_q^2$. Ignoring small-scale dissipation, the expectation of equilibrated energy level
is $\Ep(\K) = \sigma_q^2/2\mu$.

The wave action equation is
\beq
\frac{\dd}{\dd t} \underbrace{\tfrac{1}{2 f_0} \la |\phi|^2 \ra}_{\defn \A} = \tfrac{1}{2f_0}\la \phis \xi_\phi +
\phi \xi_\phis \ra -\tfrac{1}{f_0}\gamma \la |\phi|^2 \ra + \tfrac{1}{2f_0} \la \phis\D_\phi + \phi\D_\phis \ra\per
\label{A}
\eeq
The expectation of the work due to the white-noise forcing is $ \tfrac{1}{2}\la \phis \xi_\phi+\phi \xi_\phis \ra = \sigma_\phi^2\com$
and the expectation of the equilibrated action is $\Ep(\A) = \sigma_\phi^2/2f_0\gamma$.

The potential energy equation  is

\beq
\frac{\dd}{\dd t} \underbrace{\tfrac{\lambda^2}{4} \la |\nabla\phi|^2 \ra}_{\defn \P} = \Gamma_r + \Gamma_a
 -\tfrac{\lambda^2}{2}\gamma \la |\nabla\phi|^2 \ra - \tfrac{\lambda^2}{2} \la \lap\phis\D_\phi + \lap\phi\D_\phis \ra\per
\label{P}
\eeq

\section{A reference solution}

\begin{table}
 \begin{center}
   \caption{Details of the reference solution.}
   \label{parameters_reference}
   \begin{tabular}{ l | l | l }
     \hline
      Parameter & Description & Value \\
      \hline
      $\mathsf{N}$   & Number of modes &  512  \\
      $L_d$ & Domain size & $2\pi\times 200$ km \\
      $\sigma_q^2$ & Balanced-forcing variance & $1.45\,\,10^{-8}$ m$^2$ s$^{-3}$ \\
      $\sigma_w^2$ & Wave-forcing variance & $5.78\,\,10^{-8}$ m$^2$ s$^{-3}$ \\
      $k_f L_d/2\pi$    & Balanced-forcing wavenumber & 8 \\
      ${dk}_f L_d/2\pi$    & Balanced-forcing width &  1 \\
      $\mu$ & Linear bottom drag coefficient & $5.78\,\,10^{-8}$ s$^{-1}$ \\
      $\gamma$ & Linear wave damping coefficient & $2.31\,\,10^{-7}$ s$^{-1}$ \\
      $N$ & Buoyancy frequency &  $5\,\,10^{-3}$ s$^{-1}$\\
      $f_0$ & Coriolis frequency &  $1\,\,10^{-4}$ s$^{-1}$\\
      $\D_\phi$ & Exponential spectral filter & ---\\
      $\D_q$ & Exponential spectral filter & ---\\
   \end{tabular}
 \end{center}
\end{table}

\begin{figure}
\centering
\includegraphics[width=.925\textwidth]{figs/snapshots_pv_qg-niw_reference.png}
\caption{Snapshot of potential vorticity and wave action density for the solution
         with parameters in  table \ref{parameters_reference}. The scale of potential vorticity
         is $Q = \sigma_q/\mu^{1/2} $ and the scale of wave action density is
         $A = \sigma_w^2/f_0 \gamma$.}
        \label{snapshots_pv_qg-niw_reference}
\end{figure}

Figure \ref{snapshots_pv_qg-niw_reference} shows snapshots of potential vorticity and wave
action density for a solution with $\sigma_w^2 = 2\sigma_q^2$ and $\gamma = 4\mu$; table
\ref{parameters_reference} describes in detail the parameters of
this reference solution. The equilibrated potential vorticity in figure
\ref{snapshots_pv_qg-niw_reference}a  resembles the vorticity field
of waveless two-dimensional turbulence with its ubiquitous eddies, filaments,
and coherent structures. A main difference is that the the potential vorticity
of this wave-modified turbulence is more fine-grained (see spectrum?).

The snapshot of wave action density depicts the incoherent nature of the equilibrated
wave field, which is being scrambled by the turbulent balanced field
(figure \ref{snapshots_pv_qg-niw_reference}b). The wave field develops scales smaller
than the balanced eddies due to straining by the flow and wave interference.
This snapshot resembles the wave field in decaying wave-modified
two-dimensional turbulence (RWY).


\begin{figure}
\centering
\includegraphics[width=.825\textwidth]{figs/energies_reference.png}
\caption{(a) Balanced kinetic energy ($\K$) and wave potential energy ($\P$) and wave
         kinetic energy ($f_0 \A$)  for the solution with parameters in table
         \ref{parameters_reference}. The energy difference, $\Delta \K$ and $\Delta \P$,
         about a time average after equilibration ($t\,\gamma \ge 5$).}
        \label{energies_reference}
\end{figure}

\begin{figure}
\centering
\includegraphics[width=.825\textwidth]{figs/K_and_P_and_A_budget_reference.png}
\caption{The budget of (a) balanced kinetic energy ($\K$), wave potential energy ($\P$),
        and (c) wave
         kinetic energy ($f_0 \A$)  for the solution with parameters in table
         \ref{parameters_reference}. The power is scaled by the work due to the
         wave forcing $W=\sigma_w^2/2$.}
        \label{energies_reference}
\end{figure}


\end{document}
